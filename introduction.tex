\section{Introduction}
Nowadays, artificial intelligence (AI) is on everyone's lips and has become one of the most prominent trending topics in public discussion. Many of the better-known models and techniques, such as neural networks, share a common trait: they operate within a well-specified problem space and are designed to identify patterns in given examples or training data.\\
Evolutionary algorithms (EAs), on the other hand, are a set of tools designed to explore vast and often poorly understood search spaces to find optimal or near-optimal solutions. These algorithms draw inspiration from the principles of natural evolution, employing mechanisms such as selection, mutation, and crossover to iteratively improve potential solutions. Given their versatility and robustness, EAs have been successfully applied to a wide range of problems across various fields. \\
To analyze the runtime behavior of these algorithms, it is essential to understand drift theorems. Drift theorems provide a framework for examining how certain stochastic processes, like those underlying EAs, evolve over time. By understanding the drift in an evolutionary algorithm, we can gain insights into its efficiency and performance. \\
This paper aims to introduce the fundamentals of evolutionary algorithms and drift theorems to readers with little prior knowledge. Additionally, a visualization tool will be presented to provide an intuitive understanding of various test functions and their expected runtime. Finally, further reading materials will be suggested for those interested in delving deeper into the topics discussed.

%  TODO: Add a brief overview of the paper structure here.
% TODO: include motivation here and concrete example of combining neural networks and evolutionary algorithms
